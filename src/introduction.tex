\section{Introduction}

\subsection{Network Edge}
\textbf{Hosts} (end systems) access the Internet through \textbf{access networks}, running network applications,
and communicating over \textbf{links}.

Wireless access network use access points to connect hosts to routers, either via wireless LANs, e.g. Wi-Fi,
or wide-area wireless access, e.g. 4G.

Hosts can connect directly to an access network physically via guided media, e.g. twisted pair cables and fiber optic cables,
or over-the-air via unguided meia, e.g. radio.

\subsection{Network Core}
\emph{A mesh of interconnected routers which forward data in a network.}

Transmitting data through a network takes place via \textbf{circuit switching} or \textbf{packet switching}.

\subsubsection{Circuit Switching}
Circuits along the path are reserved before transmission can begin, which mean that
no other circuit can use the same path, but performance can be guaranteed.

However, there is a finite number of circuits, so the network is limited in its capacity.
This approach is used in telephone networks.

\subsubsection{Packet Switching}
Messages are broken into smaller chunks, called \textbf{packets}.
Packets are transmitted onto a link at a \textbf{transmission rate}, also known as \textbf{link capacity} or \textbf{bandwidth}.

The \textbf{packet transmission delay} $\left(d_{\text{trans}}\right)$ is the time needed to transmit an $L$-bit packet into the link 
at a transmission rate $R$.\\
\begin{equation*}
    d_{\text{trans}} = \frac{L \text{ in bits}}{R \text{ in bits/sec}} \text{ seconds}
\end{equation*}

Packets are passed from one \textbf{router} to the next across links on the path from the source to the destination.

This incurs a \textbf{propagation delay} $\left(d_{\text{prop}}\right)$, which depends on the length $d$ of the physical link,
and the propagation speed $s$ in the medium.\\
\begin{equation*}
    d_{\text{prop}} = \frac{d}{s \approx 2 \times 10^8 \; m/s}
\end{equation*}

At each router, packets are \textbf{stored and forwarded}, which means an entire packet must arrive before being
transmitted onto the next link.

Therefore, with $P$ packets and $N$ routers, the \textbf{end-to-end delay}:\\
\begin{equation*}
    d_{\text{end-to-end}} = (P + N - 1) \cdot \frac{L}{R}
\end{equation*}

At the router, packets are checked for bit errors and the output link is determined using \textbf{routing algorithms}.
This incurs a \textbf{nodal processing delay} $\left(d_{\text{proc}}\right)$.

Therefore, packets have to \textbf{queue} in a \textbf{buffer} at each router,
also incurring a \textbf{queueing delay} $\left(d_{\text{queue}}\right)$, which is the time spent waiting in the
queue before transmission.

Router buffers have a finite capacity and packets arriving to a full queue will be \textbf{dropped}.

In general,\\
\begin{equation*}
    d_{\text{end-to-end}} = d_{\text{trans}} + d_{\text{prop}} + d_{\text{queue}} + d_{\text{proc}} 
\end{equation*}

\subsubsection{Throughput}
\emph{The number of bits that can be transmitted the per unit time.}

Each link has its own \textbf{bandwidth} $R$, so throughput is measured for end-to-end communication.\\
\begin{equation*}
    \text{throughput} = \frac{1}{\sum_{i=1}^{n} \frac{1}{R_{i}}} \text{ where $n$ is the number of links}
\end{equation*}

Peak throughput and other throughput calculations are not covered in this module.

\subsection{Network Protocols}
\emph{The format and order of messages exchanged, and the actions taken after messages are sent and received.}

The protocols in the Internet are arranged in a stack of \textbf{5 layers}:
\begin{enumerate}
    \item \textbf{application}, e.g. HTTP, SMTP
    \item \textbf{transport}, e.g. TCP, UDP
    \item \textbf{network}, e.g. IP
    \item \textbf{link}, e.g. ethernet, 802.11
    \item \textbf{physical}, e.g. bits on the wire
\end{enumerate}