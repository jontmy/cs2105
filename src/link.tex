\part{Link Layer}

The \textbf{link layer} sends datagrams between \textit{adjacent} nodes
(hosts or routers) over a single link.

IP datagrams are encapsulated in \textbf{frames}.

The link layer is implemented in hardware adapters (network interface cards) or on a chip.


\section{Error Detection and Correction}
Links may be \textbf{error-prone}.

\textbf{Error detection and correction bits} (\code{EDC}) are appended to the \textbf{data bits} (\code{D})
before transmission across a link.

However, \textbf{error detection schemes} are not 100\% reliable --- they may not detect all errors,
but a larger \code{EDC} increases the probability of detecting errors.

There are 3 popular error detection schemes:
\begin{enumerate}
    \keyitem{parity checking}{works well mathematically, but not in practice as errors are clustered together}
    \keyitem*{checksums}{used in TCP/UDP/IP}
    \keyitem*{cyclic redundancy checking}{used in the link layer}
\end{enumerate}

\subsection{Parity Checking}
\begin{defn}{1D parity checking}
    Include an additional \textbf{parity bit}.

    For \textbf{even parity}, the parity bit is set to \code{0} if the number of 1s in the data bits is even,
    and \code{1} otherwise.

    For \textbf{odd parity}, the parity bit is set to \code{1} if the number of 1s in the data bits is even,
    and \code{0} otherwise.
\end{defn}

\begin{defn}{2D parity checking}
    If the data bits are arranged in a \textbf{2D matrix} with $i$ rows and $j$ columns,
    then we can compute each row and column parity.

    In addition, we can compute the parity bit for the column and row parity bits,
    for a total of $i + j + 1$ parity bits.
\end{defn}

1D parity checking can \textit{detect} \textbf{single bit errors}
(or any odd number of them), but not correct them.

2D parity checking can \textit{detect and correct} \textbf{single bit errors},
by intersecting the error row and column.

In addition, it can \textit{detect} \textbf{two-bit errors}, but not correct them.

\subsection{Cyclic Redundancy Checking (CRC)}
\begin{defn}{non-binary cyclic redundancy checking}
    First, some terminology:
    \begin{itemize}
        \keyitem*{\code{D}}{a non-binary number to transfer}
        \keyitem*{\code{R}}{an $r$-digit error detection code}
        \keyitem*{\code{G}}{an $r$-digit \textbf{generator} known to both sender and receiver}
    \end{itemize}

    To transmit the new message \code{M}:
    \begin{enumerate}
        \item Create a new number \code{D'} by appending \code{9}, $r$ times, to \code{D}.
        \item Find the remainder $y$ of \code{D'} divided by \code{G}.
        \item Transmit \code{M = D' - y}.
    \end{enumerate}

    Notice how \code{M} is divisible by \code{G} --- the receiver can calculate the new remainder $y'$,
    and discard the message if $y' \neq 0$.
\end{defn}

\begin{defn}{binary cyclic redundancy checking}
    First, some terminology:
    \begin{itemize}
        \keyitem*{\code{D}}{binary data bits}
        \keyitem*{\code{R}}{an $r$-bit error detection code}
        \keyitem*{\code{G}}{an $r + 1$-bit \textbf{generator} known to both sender and receiver}
    \end{itemize}

    All calculations are done \textbf{modulo 2}, to avoid carries for addition and borrows for subtraction
    --- now identical to \code{XOR}.

    To transmit the new message \code{M}:
    \begin{enumerate}
        \item Create a new number \code{D'} by appending \code{0}, $r$ times, to \code{D}.
        \item Find the remainder of \code{D'} divided by \code{G}, which is used as \code{R}.
        \item Transmit \code{M = (D, R)}, which is \code{R} appended to \code{D}.
    \end{enumerate}

    Now, the receiver divides \code{M} by \code{G} and checks if the remainder is \code{0}.

    A non-zero remainder indicates an error.
\end{defn}

CRC's error detection capabilities:
\begin{itemize}
    \keyitem*{single bit errors}{all odd numbers of errors}
    \keyitem*{burst errors < \code{r+1}-bits}{all such errors}
    \keyitem*{burst errors > \code{r}-bits}{probability $1- 0.5^r$}
\end{itemize}

CRC is also easy to implement on hardware due to the modulo-2 arithmetic.


\section{Link Access Control}
\textbf{Point-to-point links} connect a sender and receiver directly ---
there is no need for multiple access control.

\textbf{Broadcast links} connect multiple nodes to a single shared broadcast channel.

Every node receives a copy of every broadcasted link, causing \textbf{collision}
if two signals are received simultaneously.

\begin{defn}{ideal multiple access control}
    Given a broadcast channel of rate $R$ bps, the muliple access control protocol should be:
    \begin{enumerate}
        \item \textbf{collision-free}
        \keyitem*{efficient}{a single transmitting node should send at rate $R$}
        \keyitem*{fair}{each transmitting node among $M$ nodes should send at an average rate $R / M$}
        \keyitem*{decentralized}{no coordination between nodes}
    \end{enumerate}

    In addition, \textbf{channel sharing coordination} \textit{must use the channel itself}
    --- no other out-of-band channel.
\end{defn}

\subsection{Channel Partitioning Protocols}

\begin{defn}{time division multiple access (TDMA)}
    Let there be $n$ nodes.

    Each node is equally allocated a \textbf{time slot} of length $T$,
    spanning a \textbf{time frame} of length $n \cdot T$, during which they get access
    to the channel; outside of which they are \textbf{idle}.
\end{defn}

\begin{defn}{frequency division multiple access (FDMA)}
    FDMA is akin to TDMA but with \textbf{frequency bands} instead of time slots.
\end{defn}

TDMA and FDMA are \textit{collision-free}, \textit{perfectly fair}, \textit{decentralized},
but \textbf{inefficient} as unusued slots are wasted.

\subsection{Controlled Access Protocols}

\begin{defn}{polling protocol}
    One node is designated as the \textbf{master node}.

    The master node polls each \textbf{slave node} in turn, allowing it to transmit
    a pre-determined maximum number of frames.
\end{defn}

Polling is \textit{collision-free}, \textit{efficient}, \textit{perfectly fair},
but \textbf{not decentralized} as the master node results in a single point of failure.

\begin{defn}{token passing protocol}
    In a ring network topology, a special \textbf{token} frame, is passed between
    nodes sequentially.

    The node possessing the token can transmit a pre-determined maximum number of frames
    before forwarding the token.
\end{defn}

Token-passing is \textit{collision-free}, \textit{efficient}, \textit{perfectly fair}, and \textit{decentralized}.

However, a \textbf{token loss} and a single \textbf{node failure} can be disruptive.

\subsection{Random Access Protocols}
A node with data to send should be able to transmit at full channel data rate $R$,
with \textbf{no \textit{a priori} coordination}.

Random access protocols must \textit{detect} and \textit{recover} from collisions.

\begin{defn}{slotted ALOHA}
    Like TDMA, time is divided into slots of length $L / R$, and synchronized at each node.

    Nodes transmit only at the beginning of a slot, re-transmitting in the event of a collision
    in each subsequent slot with probability $p$.
\end{defn}

Slotted ALOHA is \textit{fair} and \textit{decentralized}, but \textbf{not collision free}.

It is also \textbf{efficient-by-definition} when there is only 1 node, 
but only 37\% efficient otherwise, as slots are wasted by collision and by being empty.

\begin{defn}{pure (unslotted) ALOHA}
    No synchonization and time slots are needed --- nodes transmit immediately at any time.

    In the event of a collision, they wait for a 1-frame transmission time 
    before re-transmitting with probability $p$.
\end{defn}

Unslotted ALOHA is worse than slotted ALOHA with 18\% maximum efficiency as collision probability
is higher.

In general, ALOHA is flawed, as transmission decision is made independently of other nodes.

\begin{defn}{carrier sense multiple access (CSMA)}
    Nodes defer transmission if the channel is busy, and transmit immediately if it is idle.
\end{defn}

Collisions still occur due to propagation delay.

Both ALOHA and CDMA are flawed, in that they do not stop transmitting when collision is detected.

\begin{defn}{CSMA/CD (collision detection)}
    Like CSMA, but abort transmission if collision is detected, and re-transmit after a random delay,
    determined by \textbf{binary exponential backoff}:

    \begin{enumerate}
        \item after a collision, choose $k \in \{0, 1\}$
        \item wait $k$ time units before re-transmitting
        \item after another collision, choose $k' \in \{0, 1, 2, 2^2 - 1\}$
        \item wait $k'$ time units before re-transmitting
        \item after $m$ collisions, choose $k'' \in \{0, 1, 2, 2^2 - 1, \dots, 2^m - 1\}$
        \item wait $k''$ time units before re-transmitting
    \end{enumerate}

    The time unit is 512-bit transmission time for Ethernet.

    A \textbf{minimum frame size} (64 bytes for Ethernet) is imposed to ensure increase
    the chance that collision is detected.
\end{defn}

Both CSMA and CSMA/CD \textit{efficient}, \textit{fair} and \textit{decentralized},
but \textbf{not collision-free}.


\section{Media Access Control (MAC)}
All nodes receive all frames in the same broadcast channel.

Thus, adapters filter received frames by their destination MAC address,
passing only those addressed to itself to the network layer.

\begin{defn}{MAC addresses}
    Every adapter has a \textbf{MAC address} --- a \textbf{48 bit hexadecimal sequence} 
    burned into ROM.

    The \textbf{first 24 bits} are allocated by IEEE and reserved for vendor identification.

    The \textbf{broadcast address} is \code{FF-FF-FF-FF-FF-FF}.

    Use \code{ifconfig} to find the MAC address of an adapter.
\end{defn}


\section{Ethernet}
\textbf{Local area network (LAN)} is a computer network within a geographical area.

\textbf{Wi-Fi} (IEEE 802.11) and \textbf{Ethernet} (IEEE 802.3) are the most common LAN technologies.

\begin{defn}{Ethernet frame structure}
    \begin{tblr}{hlines, vlines}
        preamble & \code{dst} & \code{src} & type & data... & \code{CRC} \\
        8B & 6B & 6B & 2B & 46-1500B & 4B
    \end{tblr}
\end{defn}

The data size is upper bounded by the link \textbf{maximum transmission unit (MTU)},
and lower bounded to ensure collision detection.

Ethernet uses \textbf{CRC-32} using a 32-bit generator.

The type field is necessary to specify the network layer protocol, which allows
Ethernet to multiplex different network layer protocols.

\begin{defn}{preamble sequence}
    Each frame has a \textbf{preamble} starting with 7 bytes of \code{10101010}, or $\code{AA}_\code{hex}$,
    which is used as a wake-up signal.

    The last byte in the preamble is \code{10101011}, or $\code{AB}_\code{hex}$, which
    serves as a delimiter indicating the start of the frame.

    The square wave pattern of the first 7 bytes synchronize the sender's and receiver's clock rates.
\end{defn}

Ethernet is \textbf{unreliable} as no acknowledgements are sent --- dropped frames are recovered
only by a higher layer protocol.

Ethernet uses \textbf{CSMA/CD} with exponential backoff.

\subsection{Topology}

\begin{defn}{bus topology}
    A single backbone coaxial cable to interconnect all nodes
    in a \textbf{broadcast LAN} with several drawbacks:

    \begin{enumerate}
        \item backbone cable is a single point of failure,
        \item difficult to troubleshoot,
        \item slow and not ideal for larger networks due to collision
    \end{enumerate}
\end{defn}

\begin{defn}{star topology}
    Nodes are connected to either \textbf{hubs} or \textbf{switches}.
    
    A hub is a \textbf{physical layer device} which re-creates the \textit{bits} received,
    and boosts their signal to all nodes, but very slow and not ideal for larger networks.

    A switch \textbf{link layer device} which stores and forwards \textit{frames} with no collisions,
    and are much faster than hubs.
\end{defn}

\subsection{Ethernet Switches}
Ethernet switches employ \textbf{selective forwarding} to forward frames only to single or multiple
outgoing links via \textbf{CSMA/CD} depending on the destination MAC address.

Switches have other benefits:
\begin{itemize*}
    \keyitem{transparency}{no frames are addressed to the switch, hosts remain unaware of their presence}
    \keyitem{plug-and-play}{no configuration needed}
    \keyitem{simultaneous transmission}{every node has a dedicated, direct, and \textit{buffered} link \textbf{interface} using Ethernet to the switch --- no collisions!}
    \keyitem{interconnectivity}{switches can be connected in a heirarchy to form a LAN, with a router connected to the root switch}
\end{itemize*}

\begin{defn}{selective forwarding}
    A switch knows that a host is reachable via an interface using a \textbf{switch table}.

    Every entry consists of the \textbf{host MAC address}, \textbf{interface number}, and \textbf{TTL}.

    To \textit{populate} the switch table, it has to employ \textbf{self-learning} by recording entries for every incoming frame.

    To \textit{forward} a frame to a \textit{known} host, the switch looks up the destination MAC address in the switch table,
    and forwards the frame to the corresponding interface: (\textbf{forwarding}).

    To forward a frame to an \textit{unknown} host, the switch broadcasts the frame to all interfaces except the incoming one: (\textbf{flooding}).

    There is one exception: a switch receiving a frame on an interface that it would forward it back onto is simply dropped: (\textbf{filtering}).
\end{defn}


\section{Address Resolution Protocol (ARP)}

\textbf{ARP} is used to determine the MAC address of a host given its IP address.

\begin{defn}{ARP tables}
    Every IP node maintains an \textbf{ARP table} which maps IP addresses to MAC addresses
    within a subnet, and can be listed with \code{arp}.

    Every entry has a TTL, usually in the order of minutes.
\end{defn}

ARP is \textbf{plug-and-play} --- nodes create their ARP tables without intervention from
a network administrator.

\begin{defn}{ARP queries}
    If \code{A} does not have \code{B}'s MAC address in its ARP table, it broadcasts an \textbf{ARP query
    packet} containing \code{B}'s IP address to the subnet.

    This query packet has the destination MAC address set to the broadcast address \code{FF-FF-FF-FF-FF-FF}.

    All nodes in the subnet receive the query, but \textit{only} \code{B} responds with its MAC address
    by sending it as a \textbf{reply frame}.
\end{defn}

ARP queries are used to determine destination MAC addresses if they aren't already cached in the ARP table,
and used when sending frames in the \textit{same subnet}.

\begin{defn}{sending frames outside of a subnet}
    Suppose host \code{A} wants to send a frame to host \code{B} in a different subnet, via a router \code{R}.

    \code{A} will send the frame to \code{R} with the destination MAC address of the \textit{receiving interface}
    at \code{R}.

    The IP datagram will have the destination IP address of \code{B}.

    At \code{R}, it will create the link-layer frame with the destination MAC address of \code{B} and
    source MAC address of the \textit{transmitting interface} at \code{R}.

    The Ethernet frames change between links, but the IP datagrams do not, as IP is end-to-end.
\end{defn}