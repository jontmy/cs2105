\part{Transport Layer}

Transport layer services deliver messages between processes on different hosts,
while packet switches check the destination IP address for routing.

The \textbf{sender} breaks messages into \textbf{segments}, and passes them to
the \textit{network layer}.

The \textbf{receiver} reassembles the segments into the message, and passes it
the \textit{application layer}.


\section{User Datagram Protocol (UDP)}
UDP transmission is \textbf{unreliable} but fast, often used only by loss-tolerant and rate-sensitive
applications, e.g. multimedia.

UDP adds the following on top of IP:

\begin{itemize}
    \item \textbf{multiplexing} at sender,
    \item \textbf{de-multiplexing} at receiver, and
    \item \textbf{checksum} at sender and receiver.
\end{itemize}

At the receiver, every datagram with the same \textit{destination port number}
will be directed to the \textit{same UDP socket}.

UDP adds the following headers to each segment:

\begin{enumerate}[leftmargin=*]
    \item \textbf{source port number},
    \item \textbf{destination port number},
    \item \textbf{segment length}, and
    \item \textbf{checksum}.
\end{enumerate}

UDP exists because it is:
\begin{itemize}
    \item \textbf{fast}: no connection establishment $\implies$ no delay,
    \item \textbf{simple}: no connection state,
    \item \textbf{small}: header size is small,
    \item \textbf{unlimited}: no congestion control.
\end{itemize}

\subsection{UDP Checksum}
Checksums are used to detect transmission errors in each segment, e.g. flipped bits.

Steps to compute the checksum:
\begin{enumerate}[leftmargin=*]
    \item Partition the segment into 16-bit integers.
    \item Perform \textbf{binary addition} with every integer.
    \item Add any \textit{carry bits} to the next integer.
    \item Take the 1s complement of the final integer to get the checksum.
\end{enumerate}


\section{Reliable Data Transfer (RDT)}
The underlying network layer may \textit{corrupt}, \textit{drop}, \textit{re-order},
or \textit{delay} packets during transmission --- it is \textbf{inherently unreliable}.

RDT protocols aim to \textit{guarantee packet delivery} (in the order sent) and \textit{correctness}.

RDT protocols are purely hypothetical --- they only form the basis for real-life implementations.

\subsection{RDT 1.0}
We assume that the underlying channel is \textit{perfectly reliable}.

The sender sends data through the channel, and the receiver reads the data from the channel.

\subsection{RDT 2.0}
We introduce \textbf{bit errors} into the otherwise perfect underlying channel.

Now the receiver has to validate the \textbf{checksum} to \textit{detect} bit errors,
and reply with either of the following:

\begin{itemize}
    \item \textbf{acknowledgement} (ACK): if the packet was received correctly, or,
    \item \textbf{negative acknowledgement} (NAK): if the packet was corrupted, requesting re-transmission.
\end{itemize}

The sender will re-transmit the packet after receiving an NAK, \textit{or} if either
the ACK or NAK is corrupted.

However, this is a \textbf{stop-and-wait} protocol, as the sender has to wait for each acknowledgement
before sending the next packet.

\subsection{RDT 2.1}
Because ACKs can get corrupted, causing duplicate packets to be transmitted to the receiver in RDT 2.0,
we add a \textbf{sequence number} to each packet.

If the receiver identifies a duplicate packet, it simply \textit{discards} it --- not sending it
to the application.

\subsection{RDT 2.2}
We want a \textbf{NAK-free} protocol --- such that the receiver \textit{only} sends ACKs.

The NAKs are replaced by ACKs, and \textit{all} ACKs are appended with the sequence number of the last
packet that was received correctly.

If the sender receives \textit{duplicate} or \textit{corrupted} ACKs, it re-transmits the current packet.

\subsection{RDT 3.0}
We introduce \textbf{packet loss} and \textbf{delays} to the underlying channel.

These are handled via \textbf{timeouts} which trigger re-transmission if an ACK is either
corrupted, lost, or delayed.

Because re-transmission may duplicate packets, we still use the sequence number to
discard duplicate packets.

However, this has \textbf{low throughput} and \textbf{low utilization},
because a sender spends most of its time waiting for an ACK.

\subsection{Pipelined Protocols}
We can increase throughput by \textbf{pipelining} --- sending multiple yet-to-be-acknowledged
packets at a time.

The \textbf{window size} controls the number of packets sent at a time.

This increases the \textit{throughput} and \textit{utilization}, but at a cost:
\begin{itemize}
    \item increased range of sequence numbers
    \item buffering at sender and receiver
    \item increased design complexity
\end{itemize}

\subsubsection{Go-back-\emph{n} (GBN)}
A GBN sender tracks 2 things:
\begin{enumerate}
    \item a \textbf{sliding window} of $n$ contiguous packets, and,
    \item a \textbf{timer} for the oldest unACK'd packet, which is re-transmitted in the window when it expires.
\end{enumerate}

A GBN receiver demands that all packets are received in order:
\begin{itemize}
    \item \textbf{ACK} every successive packet that is received in order
    \item \textbf{discard} every out-of-order packet (i.e. if packet $l$ is lost, every packet $> l$ is discarded until packet $l$ is received correctly)
\end{itemize}

GBN is also known as \textbf{cumulative ACK} --- an ACK means that every
packet up to that point has been received correctly.

GBN behaves exactly like RDT 3.0 when $n = 1$.

The timeout has to be chosen correctly --- too short and it results in \textbf{premature timeouts},
too large and the protocol becomes slow.

\subsubsection{Selective Repeat (SR)}
SR buffers out-of-order packets instead of dropping them.

An SR receiver ACKs individual packets.

An SR sender maintains a timer for each unACK'd packet, re-sending \textit{only that packet} 
when its timer expires.

However, the sliding window of the sender cannot advance past any unACK'd packet until
it is ACK'd.


\section{Transmission Control Protocol (TCP)}
TCP has several characteristics:

\begin{itemize*}
    \keyitem{point-to-point}{one sender, one receiver}
    \keyitem{connection-oriented}{handshakes before transmission}
    \keyitem{full-duplex service}{bi-directional data flow}
    \keyitem{reliable, in-order byte stream}{bytes are labelled by sequence numbers}
\end{itemize*}

Every TCP connection (socket) is identified by a tuple:
\begin{enumerate}
    \item the \textbf{source IP address},
    \item the \textbf{source port number},
    \item the \textbf{destination IP address}, and,
    \item the \textbf{destination port number}.
\end{enumerate}

The \textbf{maximum segment size} is \textbf{1460 bytes}, but this is \textit{not inclusive} of 20 bytes
for the \textbf{TCP packet headers}:

\begin{enumerate*}
    \keyitem*{source port number}{16 bits}
    \keyitem*{destination port number}{16 bits}
    \keyitem{sequence number \code{(seq)}}{32-bit byte number of the first byte in the segment}
    \keyitem{acknowledgement number \code{(ack)}}{32-bit byte number of the next expected byte}
    \keyitem{\code{ACK} bit \code{(A)}}{\code{1} if the ACK is in use, \code{0} otherwise}
    \keyitem{\code{SYN} bit \code{(S)}}{\code{1} if this packet is establishing a new connection, \code{0} otherwise}
    \keyitem{\code{FIN} bit \code{(F)}}{\code{1} if this packet is closing the connection, \code{0} otherwise}
    \keyitem*{checksum}{32 bits}
\end{enumerate*}

\subsection{Connection}
TCP employs a \textbf{3-way handshake} to \textit{establish a connection} before exchanging application data:

\begin{enumerate*}
    \keyitem{client $\to$ server}{\code{S = 1, seq = x}}
    \keyitem{server $\to$ client}{\code{S = 1, seq = y, A = 1, ack = x + 1}}
    \keyitem{client $\to$ server}{\code{A = 1, seq = x + 1, ack = y + 1, \\ data = ...}}
\end{enumerate*}

The client and server choose their \textbf{initial sequence numbers} \code{x} and \code{y}
\textit{randomly} and \textit{independently}.

To \textit{close a connection}:
\begin{enumerate*}
    \keyitem{client $\to$ server}{\code{F = 1, seq = x}}
    \keyitem{server $\to$ client}{\code{A = 1, ack = u + 1}}
    \keyitem{server $\to$ client}{\code{F = 1, seq = t}}
    \keyitem{client $\to$ server}{\code{A = 1, ack = t + 1}}
\end{enumerate*}

\subsection{Acknowledgements}
TCP is also a \textbf{cumulative ACK} protocol, but handling of out-of-order segments
is implementation-dependent.

Both the sender and receiver have a \textbf{buffer} for sent or received \textbf{segments}.

A TCP sender tracks 2 things:
\begin{enumerate*}
    \item the \textbf{next sequence number}, which is the first byte of the next segment to send, and,
    \item a \textit{single} \textbf{timer} for the oldest unACK'd packet, which is re-transmitted in the window when it expires.
\end{enumerate*}

There are 4 potential actions for a TCP receiver to take when it receives a segment:
\begin{enumerate*}
    \keyitem{in-order, first non-ACK'd segment}{delay ACK for up to 500ms for the next segment}
    \keyitem{in-order, previous segment pending ACK}{send \textit{single} cumulative ACK, ACKing both segments}
    \keyitem{out-of-order (gapped up) segment}{send \textit{duplicate} ACKs indicating the \code{seq} of the expected byte}
    \keyitem{gap-fill segment}{send ACK if the packet fills the lowest end of the gap, otherwise it is still out-of-order, so repeat above.}
\end{enumerate*}

ACK packets can also contain data (\textbf{piggy-back}), which is useful for bi-directional communication
as it halves the number of packets needed.

\subsection{Timeout}
The timeout interval is constantly computed based on the \textbf{estimated RTT}:

\begin{boxed}[l]
    $\code{RTT}_\code{est} = (1- \alpha) \ \code{RTT}_\code{est} + \alpha \ \code{RTT}_\code{sample}$ \\
    $\code{RTT}_\code{dev} = (1- \beta) \ \code{RTT}_\code{dev} + \beta \  \left| \code{RTT}_\code{sample} - \code{RTT}_\code{est} \right| $ \\
    $\code{timeout} = \code{RTT}_\code{est} + 4 \times \code{RTT}_\code{dev}$
\end{boxed}

Typically, $\alpha = 0.125$ and $\beta = 0.25$.

The timeout interval has a \textbf{safety margin} factor of 4 $\code{RTT}_\code{dev}$.

TCP also employs \textbf{fast retransmission} --- the sender resends segments immediately
after receiving 4 ACKs for the same segment, assuming that they were lost.