\part{Network Layer}


\section{IP Addresses}
\textbf{IP addresses} are \textbf{32-bit integers} used to identify a host or router.

Each host can have multiple IP addresses, each of which is associated with a single a \textbf{network interface}
that sends and receives packets.

This means that a device connected over Wi-Fi and Ethernet will have two IP addresses.

\begin{defn}{converting binary to decimal notation}
    An IP address can be written in \textbf{dotted decimal notation} where its 4 bytes
    are converted into decimal values between \code{0} and \code{255}.

    For example, \code{00000001 00000010 00000011 10000001} is written as \code{1.2.3.129}.
\end{defn}

IP addresses are assigned in 2 ways:
\begin{enumerate}
    \item manual configuration by a sysadmin, or,
    \item automatic assignment by a \textbf{DHCP} server.
\end{enumerate}

\subsection{Dynamic Host Configuration Protocol (DHCP)}
A \textbf{DHCP server} dynamically allocates IP addresses to hosts on a network.

This allows IP addresses to be \textbf{renewable}, \textbf{reusable}, and \textbf{scalable}.

DHCP runs on \textbf{UDP} ports 67 for the client and 68 for the server; chosen for its speed.

\begin{defn}{DHCP process}
    \vspace{0.75em}
    \begin{enumerate}
        \keyitem*{DHCP discover}{broadcasted by host}
        \keyitem*{DHCP offer}{response from DHCP server}
        \keyitem*{DHCP request}{from host to server}
        \keyitem*{DHCP ACK}{reply from server with the IP address}
    \end{enumerate}
\end{defn}

In the DHCP discovery message, the \textbf{arriving client} sends a broadcast using a \textbf{special IP} \code{0.0.0.0}
on port \code{68} in its \code{src} field because it has yet to be assigned a valid IP address.

It also uses the \textbf{broadcast address} \code{255.255.255.255} in its \code{dest} field,
which means all hosts on the same \textbf{subnet} will receive the message.

However, only the \textbf{DHCP servers} running on port \code{67} will respond with their \textbf{DHCP offers}.

\begin{defn}{DHCP message fields}
    \vspace{0.75em}
    \begin{enumerate}
        \keyitem*{\code{src}}{source IP address}
        \keyitem*{\code{dest}}{destination IP address}
        \keyitem*{\code{yiaddr}}{proposed IP address for the client}
        \keyitem*{\code{transaction}}{sequence number for the request}
        \keyitem*{\code{lifetime}}{duration the offer is valid for}
    \end{enumerate}
\end{defn}

\code{yiaddr} is proposed by the server during the DHCP offer, after which the arriving client
sends a DHCP request with it in its \code{yiaaddr} field as well.

Because there may be multiple DHCP servers, the arriving client will receive multiple DHCP offers.

The client has to broadcast its decision to \code{255.255.255.255} in the \code{dest} field of the DHCP request
so that all the other servers can rescind their offers.

DHCP can also provide additional network information:
\begin{enumerate}
    \item the IP address of the \textbf{default gateway} (first-hop router),
    \item the IP address of the local DNS server, and,
    \item the \textbf{network mask} 
\end{enumerate}

\subsection{Special IP Addresses}
Special IP addresses are \textit{ranges} in the form \code{a.b.c.d/x}.

The number represented by \code{x} indicates the number of bits in the \textbf{network prefix}
which are fixed to that range --- the remaining \code{32} $-$ \code{x} bits are the \textbf{host ID}.

\begin{defn}{special addresses}
    \begin{tblr}{lX}
        \textbf{\code{0.0.0.0/8}} & non-routable meta-address \\ \hline
        \textbf{\code{127.0.0.0/8}} & loopback address, aka. \textbf{\code{localhost}} \\ \hline
        \textbf{\code{10.0.0.0/8}} & \SetCell[r=3]{l} \textbf{private addresses} \\
        \textbf{\code{172.16.0.0/12}} \\
        \textbf{\code{192.168.0.0/16}} \\ \hline
        \textbf{\code{255.255.255.255/32}} & \textbf{broadcast address} \\
    \end{tblr}
\end{defn}

Private IP addresses are \textit{not globally unique}, and are used for \textit{internal networks}.

They do not require coordination with IANA or any Internet registry, but \textbf{network address translation}
is needed to use them on the public Internet.

Network prefixes identify a \textbf{subnet}, where a group of hosts are directly connected by cables.

Hosts within the same subnet can communicate without the need for a router.

\subsection{Classless Inter-domain Routing (CIDR)}
IP address assignment is not random; CIDR is the assignment strategy of IP addresses
using \textbf{heirarchical addressing}:

Routers map \textit{blocks} of IP addresses to a single next-hop router in a \textbf{routing/forwarding table},
rather than mapping each individual IP address.

IP addresses are bought from registries or rented from an ISP's address space, who in turn get them from ICANN.

The \textbf{subnet (network) prefix} of IP addresses are of arbitrary length, and are assigned by the ISP.

\begin{defn}{subnet mask}
    \textbf{Subnet masks} are 32-bit integers used to determine the subnet of an IP address.

    The subnet prefix bits are set to \code{1} and the host ID bits are set to \code{0}.

    For example, for the IP address \code{200.23.16.42/23}, the subnet mask is 
    \code{11111111 11111111 11111110 00000000}.

    Its decimal equivalent is \code{255.255.254.0}.
\end{defn}