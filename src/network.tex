\part{Network Layer}


\section{Routing}
Routing is done heirarchically through \textbf{autonomous systems} of routers and links
owned by ISPs.

Routing within an autonomous system is known as \textbf{intra-AS} routing.

There are 2 classes of routing algorithms:
\begin{enumerate*}
    \keyitem{link-state algorithms}{every router has complete knowledge of the network topology,
    as they all broadcast link costs periodically}
    \keyitem{distance-vector algorithms}{routers only know their direct neighbors and their link costs,
    and they have to exchange \textbf{local views} with neighbors and update their own}
\end{enumerate*}

\begin{defn}{abstracting network topology}
    A network can be viewed as a graph:
    \begin{itemize}
        \keyitem*{vertices}{routers}
        \keyitem*{edges}{physical links between routers}
        \keyitem*{costs}{either constant, or inversely related to bandwidth}
    \end{itemize}

    Routing involves finding the \textbf{least-cost path} between two vertices:
    \begin{itemize*}
        \keyitem{\code{c(x, y)}}{link cost between routers \code{x} and \code{y}, 
        or $\infty$ if \code{x} and \code{y} are not \textbf{direct neighbors}}
        \keyitem{\code{d}$_\code{x}$\code{(y) = min}$_\code{v}$\code{\{c(x, v) + d}$_\code{v}$\code{(y)\}}}
        {least-cost path from \code{x} to \code{y}, where \code{min} is taken over all direct neighbors of \code{x}}
    \end{itemize*}
\end{defn}

\begin{defn*}{updating local views}
    \begin{enumerate*}
        \item Every router sends its distance vectors to its direct neighbors.
        \item If a router receives a distance vector from a neighbor, and finds a shorter path to another router, it updates its own local view.
        \item After several iterations, all routers will have the same local view.
    \end{enumerate*}
\end{defn*}

\subsection{Routing Information Protocol (RIP)}
RIP implements the distance-vector algorithm, using \textbf{hop count} as the cost metric,
and runs over UDP every 30 seconds, with \textbf{self-repair} assuming that
a neighbor without an update after 3 seconds has failed.


\section{Internet Protocol (IPv4)}

\subsection{IP Addresses}
\textbf{IP addresses} are \textbf{32-bit integers} used to identify a host or router.

Each host can have multiple IP addresses, each of which is associated with a single a \textbf{network interface}
that sends and receives packets.

This means that a device connected over Wi-Fi and Ethernet will have two IP addresses.

\begin{defn}{converting binary to decimal notation}
    An IP address can be written in \textbf{dotted decimal notation} where its 4 bytes
    are converted into decimal values between \code{0} and \code{255}.

    For example, \code{00000001 00000010 00000011 10000001} is written as \code{1.2.3.129}.
\end{defn}

IP addresses are assigned in 2 ways:
\begin{enumerate}
    \item manual configuration by a sysadmin, or,
    \item automatic assignment by a \textbf{DHCP} server.
\end{enumerate}

\subsubsection{Dynamic Host Configuration Protocol (DHCP)}
A \textbf{DHCP server} dynamically allocates IP addresses to hosts on a network.

This allows IP addresses to be \textbf{renewable}, \textbf{reusable}, and \textbf{scalable}.

DHCP runs on \textbf{UDP} ports 67 for the client and 68 for the server; chosen for its speed.

\begin{defn}{DHCP process}
    \vspace{0.75em}
    \begin{enumerate}
        \keyitem*{DHCP discover}{broadcasted by host}
        \keyitem*{DHCP offer}{response from DHCP server}
        \keyitem*{DHCP request}{from host to server}
        \keyitem*{DHCP ACK}{reply from server with the IP address}
    \end{enumerate}
\end{defn}

In the DHCP discovery message, the \textbf{arriving client} sends a broadcast using a \textbf{special IP} \code{0.0.0.0}
on port \code{68} in its \code{src} field because it has yet to be assigned a valid IP address.

It also uses the \textbf{broadcast address} \code{255.255.255.255} in its \code{dest} field,
which means all hosts on the same \textbf{subnet} will receive the message.

However, only the \textbf{DHCP servers} running on port \code{67} will respond with their \textbf{DHCP offers}.

\begin{defn}{DHCP message fields}
    \vspace{0.75em}
    \begin{enumerate}
        \keyitem*{\code{src}}{source IP address}
        \keyitem*{\code{dest}}{destination IP address}
        \keyitem*{\code{yiaddr}}{proposed IP address for the client}
        \keyitem*{\code{transaction}}{sequence number for the request}
        \keyitem*{\code{lifetime}}{duration the offer is valid for}
    \end{enumerate}
\end{defn}

\code{yiaddr} is proposed by the server during the DHCP offer, after which the arriving client
sends a DHCP request with it in its \code{yiaaddr} field as well.

Because there may be multiple DHCP servers, the arriving client will receive multiple DHCP offers.

The client has to broadcast its decision to \code{255.255.255.255} in the \code{dest} field of the DHCP request
so that all the other servers can rescind their offers.

DHCP can also provide additional network information:
\begin{enumerate}
    \item the IP address of the \textbf{default gateway} (first-hop router),
    \item the IP address of the local DNS server, and,
    \item the \textbf{network mask} 
\end{enumerate}

\subsubsection{Special IP Addresses}
Special IP addresses are \textit{ranges} in the form \code{a.b.c.d/x}.

The number represented by \code{x} indicates the number of bits in the \textbf{network prefix}
which are fixed to that range --- the remaining \code{32} $-$ \code{x} bits are the \textbf{host ID}.

\begin{defn}{special addresses}
    \begin{tblr}{lX}
        \textbf{\code{0.0.0.0/8}} & non-routable meta-address \\ \hline
        \textbf{\code{127.0.0.0/8}} & loopback address, aka. \textbf{\code{localhost}} \\ \hline
        \textbf{\code{10.0.0.0/8}} & \SetCell[r=3]{l} \textbf{private addresses} \\
        \textbf{\code{172.16.0.0/12}} \\
        \textbf{\code{192.168.0.0/16}} \\ \hline
        \textbf{\code{255.255.255.255/32}} & \textbf{broadcast address} \\
    \end{tblr}
\end{defn}

Private IP addresses are \textit{not globally unique}, and are used for \textit{internal networks}.

They do not require coordination with IANA or any Internet registry, but \textbf{network address translation}
is needed to use them on the public Internet.

Network prefixes identify a \textbf{subnet}, where a group of hosts are directly connected by cables.

Hosts within the same subnet can communicate without the need for a router.

\subsubsection{Classless Inter-domain Routing (CIDR)}
IP address assignment is not random; CIDR is the assignment strategy of IP addresses
using \textbf{heirarchical addressing}:

Routers map \textit{blocks} of IP addresses to a single next-hop router in a \textbf{routing/forwarding table},
rather than mapping each individual IP address.

IP addresses are bought from registries or rented from an ISP's address space, who in turn get them from ICANN.

The \textbf{subnet (network) prefix} of IP addresses are of arbitrary length, and are assigned by the ISP.

\begin{defn}{subnet mask}
    \textbf{Subnet masks} are 32-bit integers used to determine the subnet of an IP address.

    The subnet prefix bits are set to \code{1} and the host ID bits are set to \code{0}.

    For example, for the IP address \code{200.23.16.42/23}, the subnet mask is 
    \code{11111111 11111111 11111110 00000000}.

    Its decimal equivalent is \code{255.255.254.0}.
\end{defn}

\subsubsection{Network Address Translation (NAT)}
\textbf{Private IP addresses}, unlike public IP addresses, are not globally unique and cannot be used as
destination IP addresses for routing.

A \textbf{NAT router} bridges the gap between local networks and the Internet by replacing
private IP addresses and port numbers with its own.

It does this for all outgoing datagrams, and remembers and undoes the replacement for incoming packets
using a \textbf{NAT translation table}.


\subsection{IPv4 Headers}
\textbf{Maximum transfer unit} (MTU) is the maximum size of a datagram that can be sent over a link.

IP datagrams which are too large may be \textbf{fragmented} by routers and \textbf{reassembled} at the receiver.

\begin{defn}{IP fragmentation}
    All fragments of an IP segment have the same ID.

    The flag is set to \code{0} if it is the last fragment in the segment,
    and \code{1} otherwise.

    The offset is the number of 8-byte blocks of data (excluding the headers) from the start.
\end{defn}

Therefore, the IPv4 header is 20 bytes long:

\begin{tblr}{
    colspec = {|llX|},
    row{1} = {font=\bfseries},
    cell{4}{3} = {r=3}{l},
} \hline
    field & bits & description \\ \hline
    version & 1 & always 4 for IPv4 \\
    total length & 16 & length of the entire datagram (headers + data) \\ \hline[dashed]
    identifier & 16 & for fragmentation/reassembly \\
    flags & 3 \\
    fragment offset & 13 \\ \hline[dashed]
    time-to-live & 8 & number of remaining hops, decremented by 1 at each router, and discarded if 0 \\
    protocol & 8 & identifies the protocol used by the transport layer, e.g. TCP/UDP \\
    checksum & 16 & computed for the header only \\ \hline[dashed]
    source & 32 & sender IP address  \\ \hline[dashed]
    destination & 32 & receiver IP address \\ \hline
\end{tblr}


\section{Internet Control Message Protocol (ICMP)}
ICMP is used by hosts and routers to communicate network-level information,
such as error reports and echos.

ICMP is carried in IP datagrams and has its own set of headers which are appended after the IP headers.