\section{Application Layer}
Application layer protocols define the:
\begin{enumerate}
    \item \textbf{types of messages exchanged}, e.g. requests, responses
    \item \textbf{message syntax}, e.g. message fields and delineation
    \item \textbf{message semantics}, i.e. meaning of information in fields
    \item \textbf{rules} for when and how applications send and respond to messages
\end{enumerate}


\subsection{Architectures}
In the \textbf{client-server} architecture, a \textbf{client initiates contact} with a \textbf{server},
which waits for the request before providing a service back to the client.

This relies on data centers for scaling, and clients are usually implemented in web browsers.

In the \textbf{peer-to-peer} architecture, arbitrary end systems communicate directly with each other,
requesting and returning services.

This is \textbf{self-scalable} as new peers bring service capacity and demand.
However, this architecture is more complex as peers are connected intermittently.

Regardless of which architecture is used, the application layer ride on the \textbf{transport layer} protocols
--- \textbf{TCP} or \textbf{UDP} --- for data integrity, throughput, timing, and security.


\subsection{Hypertext Transfer Protocol (HTTP)}
\emph{The application layer protocol of the Internet.}

HTTP uses the client-server architecture and TCP as the transport service.
The client must \textbf{initiate a TCP connection} with the server before sending a \textbf{request message}.

The server receives the request message and sends the \textbf{response message} with the requested object
back to the client.

The \textbf{round-trip time} (RTT) is the time taken for a packet to travel from a client to the server and back.

The \textbf{HTTP response time} in general takes one RTT to establish the TCP connection, one more RTT for the
HTTP request to be fulfilled, plus the file transmission delay.

\subsubsection{Request Message}
\begin{lstlisting}
GET /index.html HTTP/1.1\r\n
Host: www.example.org\r\n
Connection: keep-alive\r\n
...
\r\n
<body>
\end{lstlisting}

Line 1 is the \textbf{request line}, specifying the \textbf{method}, \textbf{URL}, and \textbf{HTTP version}.

Lines 2 to 4 are the \textbf{header lines}, each specifying the \textbf{header field name} and \textbf{value}.
Only the \code{Host} header is required.

The extra blank line (line 5) indicates the end of the header lines, after which the body follows.

\subsubsection{Response Message}
\begin{lstlisting}
HTTP/1.1 200 OK\r\n
Date: Wed, 23 Jan 2019 13:11:15 GMT\r\n
Content-Length: 606\r\n
Content-Type: text/html\r\n
...
\r\n
<data>
\end{lstlisting}

Line 1 is the \textbf{status line} specifying the \textbf{HTTP version} and the \textbf{response status code}.

Lines 2 to 5 are the \textbf{header lines}, and lines 7 and onward contain the data requested, e.g. the HTML file.

\subsubsection{HTTP 1.0 (non-persistent HTTP)}
At most one object is sent over a TCP connection, after which the connection is closed.

Downloading multiple objects therefore requires multiple connections, incurring 2 RTTs per object in addition to the overhead for each TCP
connection, which some browsers may parallelize.

\subsubsection{HTTP 1.1 (persistent HTTP)}
Unlike HTTP 1.0, the server leaves the TCP connection open after sending the response, which is reused for subsequent messages.

\textbf{Persistent connections with pipelining} allow multiple objects to be requested even before the server
has responded to previous requests.

This reduces the total response time to as low as one RTT.

\subsubsection{Conditional \code{GET}}
\emph{Avoiding unnecessary requests for cached and up-to-date objects.}

Clients send an additional \code{If-Modified-Since} header with the request, containing the date of last modification.

If the requested object has been modified after the date specified, then the server responds with a \code{200 OK} along
with the requested object data.

Otherwise, the server responds with a \code{304 Not Modified}, which means the client can use its cached version.

\subsubsection{Cookies}
\emph{Maintaining state despite the stateless nature of HTTP.}

Cookies are sent using the \code{Cookie} and \code{Set-Cookie} header fields in requests and responses respectively.

They are created by servers, stored client-side, and managed by browsers.

Cookies are sent to the server in subsequent requests,
which the server can then use to execute \textbf{cookie-specific actions}, e.g. retrieve a shopping cart.