\part{Network Security}

The tenets of network security are:
\begin{itemize*}
    \keyitem*{confidentiality}{only the sender and intended receiver should understand message contents}
    \keyitem*{authentication}{sender and receiver want to confirm the other's identity}
    \keyitem*{message integrity}{messages should not be altered without detection}
    \keyitem*{access and availability}{services must be accessible and available to users}
\end{itemize*}

\section{Cryptography}
Cryptographic techniques allow senders to encrypt data (\textbf{ciphertext}) and receivers 
to recover the original data (\textbf{plaintext}).

A \textbf{key} is a string of input numbers or characters to the encryption or decryption algorithm:
\begin{itemize*}
    \keyitem*{$m$}{plaintext message}
    \keyitem*{$K_A()$}{encryption algorithm with key $K_A$, such that $K_A(m)$ is the ciphertext}
    \keyitem*{$K_B()$}{decryption algorithm with key $K_B$, such that $K_B(K_A(m)) = m$}
\end{itemize*}

\subsection{Symmetric Key Cryptography}

\textbf{Symmetric key cryptography} uses the same key $K_S$ for encryption and decryption ($K_A = K_B$),
but the algorithms $K_A()$ and $K_B$ need not be the same.

However, the need for an agreed, shared key is a major drawback of symmetric key cryptography.

\begin{defn}{simple substitution ciphers}
    In a \textbf{Caesar's cipher}, each alphabet is shifted by a fixed number ---
    only 25 possible shift values, which is easy to brute-force.

    In a \textbf{monoalphabetic cipher}, each alphabet is mapped to a different alphabet,
    e.g. by a random permutation --- vulnerable to statistical analysis and common $n$-grams.

    In a \textbf{polyalphabetic cipher}, $n$ ciphers are rotated in a predefined sequence known as a \textbf{cycling pattern}.
\end{defn}

Encryption schemes can be broken in 3 ways:
\begin{enumerate}
    \keyitem*{ciphertext-only attack}{only ciphertext is known, and can be analyzed}
    \keyitem*{known-plaintext attack}{ciphertext is \textit{known} for a corresponding plaintext}
    \keyitem*{chosen-plaintext attack}{ciphertext can be \textit{requested} for a corresponding ciphertext}
\end{enumerate}

\begin{defn}{block ciphers}
    Messages are encrypted in blocks of $K$ bits, and each block is encrypted independently.

    Blocks are mapped by a one-to-one substitution to a different block, such that there are $2^K!$ keys.
\end{defn}

\textbf{Data Encryption Standard (DES)} is a block cipher with 56-bit keys and 64-bit blocks,
but is vulnerable to brute-force attacks.
 
\textbf{Advanced Encryption Standard (AES)} replaced DES, and uses 128 bit blocks and 128, 192, or 256 bit keys.

\subsection{Asymmetric Key Cryptography}
Also known as \textbf{public key cryptography}, using different keys for encryption and decryption:

\begin{itemize}
    \keyitem*{public key $K_B^+$}{known to everyone, used for encryption}
    \keyitem*{private key $K_B^-$}{known only to the receiver, used for decryption}
\end{itemize}

Similar to symmetric key cryptography, the plaintext $m = K_B^-(K_B^+(m))$.

However, it must be impossible to compute the private key $K_B^-$ from the public key $K_B^+$.

\begin{defn}{key property of modular arithmetic}
    $(a \mod{n})^d \mod{n} = a^d \mod{n}$
\end{defn}

\subsubsection{RSA}
A \textbf{message} is a bit pattern that can be uniquely represented by an integer,
such that encrypting a message is equivalent to encrypting a number.

\begin{defn*}{creating RSA public/private key pairs}
    \begin{enumerate}
        \item choose two large prime numbers $p$ and $q$
        \item compute $n = p \cdot q$, $z = (p - 1)(q - 1)$
        \item choose $e$ such that $e$ and $z$ are coprime (only common factor is 1), and $e < n$.
        \item choose $d$ such that $e \cdot d \mod{z} = 1$ 
    \end{enumerate}
    The public key $K_B^+ = (n, e)$ and the private key $K_B^- = (n, d)$.
\end{defn*}

\begin{defn}{RSA encryption and decryption}
    The ciphertext $c = m^e \mod{n}$ is decrypted into \\ $m = c^d \mod{n}$, where $m$ is the message.
\end{defn}

In RSA, the order of encryption and decryption is irrelevant: $K_B^+(K_B^-(m)) = m = K_B^-(K_B^+(m))$.

In practice, RSA is used only to transfer a symmetric \textbf{session key} $K_S$ which is then used
as the symmetric key in DES to encrypt data in a session.

This is because exponentiation in RSA is computationally expensive, and DES is significantly faster.

\subsection{Hashing}
Checksums and CRC are useful for detecting errors, but are vulnerable to attacks as they are not
collision-resistant.

A \textbf{hash function} $H()$ takes a message $m$ and returns a \textit{fixed-length} message digest
(\textbf{fingerprint}) $h$.

A \textbf{cryptographic hash function} is a hash function such that it is computationally infeasible
to find a collision, i.e. $H(m_1) = H(m_2)$ for $m_1 \neq m_2$.

\textbf{MD5} and \textbf{SHA-1} have been \textbf{cryptographically broken},
and are replaced by \textbf{SHA-2} and \textbf{SHA-3}.

\subsubsection{Hash Functions}
MD5 produces a 128 bit fingerprint using the \code{md5sum} command at the command line.

A small change in the message $m$ results in a large change in the message digest $H(m)$.

However, this still does not preserve message integrity as an attacker can simply replace
a message and its hash $(m, H(s))$ with their own $(m', H(s'))$.

\subsubsection{Message Authentication Code}
Messages are sent as $(m, H(m + s))$, where $s$ is an \textbf{authentication key}
known only to the sender and receiver.

\subsection{Digital Signatures}
Digital signatures must be \textbf{verifiable} and \textbf{unforgeable}.

\begin{defn*}{digital signing}
    \begin{enumerate*}
        \item $B$ signs $m$ by encrypting with private key $K_B^-$, producing the signature $K_B^-(m)$
        \item the message is paired with its signature and sent as $(m, K_B^-(m))$
        \item the signature $K_B^-(m)$ is verified by decrypting with public key $K_B^+$, producing $K_B^+(K_B^-(m))$
    \end{enumerate*}
    $K_B^+(K_B^-(m)) = m \implies m$ is verified to have been signed by $B$, as whoever signed $m$
    must have used $B$'s private key, $K_B^-$.
\end{defn*}

Optimization of digital signing involves signing the \textit{hash} of the message instead
of the entire message, and the hashes $H(m)$ and $K_B^+(K_B^-(H(m)))$ are compared at the receiver.

\subsubsection{Public Key Certification}
An attacker can sign a message with their own private key, and send it with their own public key
and claiming to be someone else.

Therefore, to verify that a public key belongs to a particular person, a \textbf{certificate authority} (CA)
maintains a public database of public keys and their corresponding identities.

However, an attacker can still intercept communication with the CA and alter it,
so the CA signs its messages.

However, we still do not know the CA's public key, so such knowledge is made universal
at the OS level as a list of \textbf{trusted root certification authorities}.

\begin{defn}{certification}
    Any entity $E$, e.g. person or router can prove its identity to the CA.

    The CA creates a \textbf{certificate} binding $E$ to its public key $K_E^+$, and signs it with its private key $K_{CA}^-$.

    This certificate can be verified by decrypting with the CA's public key $K_{CA}^+$, producing $K_{CA}^+(K_{CA}^-(K_E^+)) = K_E^+$.
\end{defn}


\section{Firewalls}
\textbf{Firewalls} isolate an internal network from the Internet, and selectively allow
packets to pass through.

This can prevent \textbf{denial-of-service} (DoS) attacks and illegal modification or access
of internal data.

This can also allow \textit{only} authorized users to access the internal network.

However, they are vulnerable to \textbf{IP spoofing} and can become a bottleneck.

\begin{defn}{stateless packet filtering}
    A \textbf{router firewall} filters packet-by-packet and forwards or drops packets based on any of the following:
    \begin{itemize}
        \item source and destination IP addresses
        \item TCP/UDP source and destination port numbers
        \item ICMP message type
        \item TCP \code{SYN} and \code{ACK} bits (prevents \textbf{\code{SYN} flooding})
    \end{itemize}
\end{defn}

\textbf{Access control lists} are a table of rules applied sequentially from top to bottom on incoming packets
to determine whether to allow or deny a packet.